\chapter{Introduction}

Epex Spot SE 
The European Power Exchange (EPEX SPOT SE) is the leading exchange for short-term electricity wholesale in the European Union. Founded in 2008 by the merger of the energy exchanges Powernext SA in France and EEX AG in Germany it provides a spot market for transnational day-ahead and intraday electricity supply and demand. EPEX SPOT SE has the legal form of an European Company, Societas Europaea, with its headquarters in Paris and branches in Amsterdam, London, Leipzig, Vienna and Bern. The majority owner of the company, by holding 51 % of its shares, is the European Energy Exchange Group (EEX Group), while the remaining 49 % of the shares are being held by transmission system operators. 
EPEX SPOT SE is providing its short-term energy market services in Germany, France, Austria, Switzerland, Luxemburg and since 2015, due to a merger with APX Group, also in Belgium, the Netherlands and the United Kingdom. 
EPEX SPOT SE is a company with a two-tier governance system. The shareholders appoint a Supervisory Board, which elects the Management Board. The Management Board of EPEX SPOT SE is in charge of operating the exchange and takes commercial, economic and operational decisions. 
The core business of EPEX SPOT consists in operating a power exchange for the German, French, British, Dutch, Belgian, Austrian, Swiss and Luxembourgian markets.
EPEX SPOT provides a market place where Exchange members send their orders to buy or sell electricity in determined delivery areas and matches these orders in a transparent manner, according to the public exchange rules. As an important result of this process, EPEX SPOT broadcasts the prices resulting from the trades.
These prices serve as a benchmark for the transactions of the wholesale market and they ensure competitive prices for the end-consumers, which have the freedom to choose between numerous electricity suppliers. EPEX SPOT provides a critical liquidity outlet for producers, suppliers and transmission system operators, as well as for industrial consumers, to fulfill their sales or their purchases in short term power.
EPEX SPOT SE offers a variety of subscription plans, which grant access to their market data. The data set used in this Thesis has been accessed via the University of Innsbruck and includes renewable energy day-ahead auction data of Austria and Germany. 
Part of EEX Group
EPEX SPOT operates the power spot market of EEX Group, a group of specialised companies providing the market platform for energy and commodity products for participants in more than 30 countries worldwide. The offering of the group comprises contracts for Energy, Environmentals, Freight, Metals and Agriculturals. With high specialisation and local presence in their core markets, the companies of EEX Group answer to the needs of their customers for tailor-made solutions and easy market access. The synergetic, integrated group portfolio is completed by two clearing houses which ensure proper clearing and settlement of trading transactions.
EEX Group consists of the following companies: European Energy Exchange (EEX), the European Power Exchange (EPEX SPOT), Powernext, Cleartrade Exchange, Power Exchange Central Europe (PXE), Gaspoint Nordic, Nodal Exchange and the clearing companies European Commodity Clearing (ECC) and Nodal Clear. EEX Group is based in 16 worldwide locations and is part of Deutsche Börse Group.